%%%%%%%%%%%%%%%%%%%%%%%%%%%%%%%%%%%%%%%%%%%%%%%%%%%%%%%%%%%%%%%%%%%%%%
%     File: ExtendedAbstract_intro.tex                               %
%     Tex Master: ExtendedAbstract.tex                               %
%                                                                    %
%     Author: Andre Calado Marta                                     %
%     Last modified : 27 Dez 2011                                    %
%%%%%%%%%%%%%%%%%%%%%%%%%%%%%%%%%%%%%%%%%%%%%%%%%%%%%%%%%%%%%%%%%%%%%%
% State the objectives of the work and provide an adequate background,
% avoiding a detailed literature survey or a summary of the results.
%%%%%%%%%%%%%%%%%%%%%%%%%%%%%%%%%%%%%%%%%%%%%%%%%%%%%%%%%%%%%%%%%%%%%%

\section{Introduction}
\label{sec:intro}

In recent years there has been a sharp increase in the number of \gls{iot}
devices and this trend is expected to continue\cite{IoTnumb83:online}. The IoT is a network
of interconnected devices, which exchange data with one another over
the internet. In fact, it can be any object that has an assigned
IP address and is provided with the ability to transfer data over a network. 
While there are many types of IoT devices, all of them
are restricted: they have limited memory, processing power and
available energy. This doest not mean, however, that such devices are only
capable of running the least demanding algorithms. Various devices, with different
hardware characteristics fall under the definition of \gls{iot}.
While for some of them symmetric cryptography is the only viable option,
others have resources that allow them to use public key cryptography.
Examples of IoT devices include temperature sensors,
smart light bulbs and physical activity trackers.

While inter-device communication has numerous benefits, it is important
to ensure the security of that communication. For example, when you log
in to your online banking account, you do not want others to be
able to see your password, as this may lead to the compromise of your
account. Having your account compromised means that a malicious entity
might take a hold of your money. Despite of all of the benefits that the \gls{iot} 
technology brings, communication security is often an afterthought and is 
frequently ignored.

\gls{tls} is one of the most used protocols for communication security. It
powers numerous technologies, such as \gls{https}. TLS offers the
security services of authentication, confidentiality, privacy, integrity, replay
protection and perfect forward secrecy. It is not a requirement to use all of
those services for every TLS connection. The protocol is similar to
a framework, in the sense that you can enable individual security
services on a per-connection basis. Foregoing unnecessary services will lead 
to a smaller resource usage, which in turn leads to smaller execution time and 
power usage. This is especially important in the context of IoT, due to the 
constrained nature of the devices. For example, while confidentiality, integrity and authentication 
are important when the device communicates with an external service, the first security property
is not crucial when the device is downloading a firmware update. In the latter case,
integrity and authentication would be enough.

While \gls{tls} was not designed for the constrained environment of \gls{iot}, it is a 
malleable protocol and can be configured to one's needs. For each security service that the protcol
offers, there is an array of algorithms that can be used to implement it.
If those algorithms are chosen properly, it is possible to use \gls{tls}
in the context of \gls{iot}.

The majority of existing work on \gls{tlsd} optimization proposes 
a solution that is either tied to a
specific protocol, such as \gls{coap}, or requires an introduction of a third-party
entity, such as the trust anchor in the case of the S3K system\cite{S3KScala62:online} or
even both. This has two main issues. First, a protocol-specific solution cannot
be easily used in an environment where (D)\gls{tls} is not used with that protocol.
Second, the requirement of a third-party
introduces additional cost and complexity, which will be a big resistance factor
in adopting the technology. This is especially true for developers working on
personal projects or projects for small businesses, leaving the communications insecure
in the worse case scenario. Therefore a solution that is protocol independent and fully 
compatible with the \gls{tlsd} standard and existing infrastructure is desired.

Another area that the existing literature fails to address is that it almost exclusively focuses on \gls{dtls} optimization
and not all of it can be applied to \gls{tls}. Herein we want to further explore \gls{tls} optimization. 
There is clearly a need for that,
especially with \gls{coap} over TCP and \gls{tls} standard\cite{I-D.ietf-core-coap-tcp-tls} being currently developed. The
aforementioned standard does not explore any \gls{tls} optimizations, and since any
\gls{iot} device using it in the future would benefit from them, this is an important
area to explore.

The objective of this work is to provide a means of assisting application developers
who wish to include secure communications in their applications to make
security/resource usage trade-offs, according to the environment's needs
and limitations. We aim to provide a general overview of of the costs of the \gls{tls} protocol as a whole and
of its individual parts. This is will allow to answer questions such as \textit{How much will we save if we use protocol X instead of Y for authentication?}.
Thus, performing evaluations on specific \gls{iot} hardware or analyzing \gls{tls}-specific optimizations
on it is outside of the scope of this paper.

In order to achieve our goals, a detailed cost evaluation of \gls{tls} is needed. 
With this information, the programer will be able to choose a configuration that
meets his security requirements and device constraints. If the limitations of the device's hardware
do not allow to meet the requirements, the programer may decide on an alternative configuration, possibly with
a loss of some security services and a lower security level, or forgo using (D)\gls{tls} altogether.
Thus, this work is targeted towards developers and InfoSec professionals who wish to add communication security
to applications in the IoT environment.

In our work, we performed a thorough cost evaluation of the \gls{tls} $1.2$ implementation in \textit{mbedTLS 2.7.0}.
\textit{mbedTLS} is among the most popular \gls{tls} implementation libraries for embedded systems. We evaluated costs 
in terms of the estimated number of CPU cycles and time taken. The time values were read directly from
the processor's registers. In our analysis we will show that the estimates do reflect real values, by comparing them to time
measurements obtained directly from the CPU registers. We evaluated every single one of the $161$ \gls{tls} configurations available 
in \textit{mbedTLS 2.7.0}, at $4$ different security levels.

A \gls{tls} connection consists of two main parts: first, the peers establish a secure communication channel in the Handshake phase,
followed by the data exchange using that channel in the Record phase. We focused on the Handshake part of the protocol for two main reasons.
First, it is the part with the most variability in terms of cost, due to the complex combinations of different possible algorithms.
Second, it is part which has been the least studied by existing work. The Record phase mainly consists in the use symmetric encryption 
algorithms and hash functions. 
Their costs has already been thoroughly studied by existing work.

Although our focus was on the Handshake, we also profiled the costs of the symmetric encryption algorithms and hash functions. 
We concluded with a typical configuration used on the internet, only a few megabytes
While the cost of the
Handshake might dominate when small amounts of data are transmitted, ? Our conclusions were that when the devices are expected to transfer a large amount of data 
(a few megabytes, for a typical configuration used on the internet), the focus should not be on
optimizing the Handshake, but rather the algorithms used to to provide data confidentiality and integrity.