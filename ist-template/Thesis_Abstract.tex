%%%%%%%%%%%%%%%%%%%%%%%%%%%%%%%%%%%%%%%%%%%%%%%%%%%%%%%%%%%%%%%%%%%%%%%%
%                                                                      %
%     File: Thesis_Abstract.tex                                        %
%     Tex Master: Thesis.tex                                           %
%                                                                      %
%     Author: Andre C. Marta                                           %
%     Last modified :  2 Jul 2015                                      %
%                                                                      %
%%%%%%%%%%%%%%%%%%%%%%%%%%%%%%%%%%%%%%%%%%%%%%%%%%%%%%%%%%%%%%%%%%%%%%%%

\section*{Abstract}

% Add entry in the table of contents as section
\addcontentsline{toc}{section}{Abstract}

\gls{tls} is one of the most used communication security protocols in the world. It comes with many configurations. 
Each configuration offers a set o security services, which has an implication on the 
security level and computational cost.
Not all of those configurations can be used with the resource constrained \gls{iot} devices, due to the
high computational and memory demands. Most of the existing work focuses on \gls{dtls}
and cannot be easily integrated with existing deployments. Existing work fails
to evaluate the cost of various \gls{tls} configurations and its security services.
This work focuses on cost analysis of the security services of the TLS protocol.
We evaluate the number of CPU cycles used and time taken by each \gls{tls} configuration
and security service. Software developers can use this information
to make security/cost trade-offs based on the environment needs and limitations.  

\vfill

\textbf{\Large Keywords:} TLS, DTLS, SSL, IoT, lightweight cryptograpy

