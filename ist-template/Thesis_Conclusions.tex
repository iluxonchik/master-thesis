%%%%%%%%%%%%%%%%%%%%%%%%%%%%%%%%%%%%%%%%%%%%%%%%%%%%%%%%%%%%%%%%%%%%%%%%
%                                                                      %
%     File: Thesis_Conclusions.tex                                     %
%     Tex Master: Thesis.tex                                           %
%                                                                      %
%     Author: Andre C. Marta                                           %
%     Last modified :  2 Jul 2015                                      %
%                                                                      %
%%%%%%%%%%%%%%%%%%%%%%%%%%%%%%%%%%%%%%%%%%%%%%%%%%%%%%%%%%%%%%%%%%%%%%%%

\chapter{Conclusions and Future Work}
\label{chapter:conclusions}

\section{Conclusions}

The lack of security in \gls{iot} is a serious issue that can lead to a high monetary costs,
especially when botnets infect the devices. Recent
attacks clearly show that serious damage can be caused \cite{sec17ant94:online}. An old saying attributed to the
\gls{nsa} states that "Attacks always get better; they never get worse".
Combined with the fact that the number of \gls{iot} devices is growing at a high
pace, without any major improvements to their security, makes it clear
that it is fundamental for this issue to be addressed.

\gls{tls} is one of the most used communication security protocols in the world. It offers the
security services of authentication, confidentiality, privacy, integrity, replay
protection and perfect forward secrecy. It is not a requirement to use all of
those services for every TLS connection. The protocol is similar to
a framework, in the sense that you can enable some of the security
services on a per-connection basis.

While \gls{tls} does have many configurations, not all of them can be used with the \gls{iot},
due to the constrained nature of the devices. In many cases it is necessary to make security/cost
trade-offs. In order to be able to make them, it is important to know the costs of each security
service. Current work offers no such information. Thus, a software developer wishing to use \gls{tls}
for connection security in a constrained environment does not have a reference to go to.

In our work, we decomposed \gls{tls} into individual parts and evaluated the cost of each security service.
We have evaluated the cost of every \gls{tls} configuration available in \textit{mbedTLS 2.7.0} and the underlying
algorithms. The herein presented results can be used to make informed decisions about the security/cost trade-offs,
specific to the environment.

First, we did a thorough analysis of \gls{tls}, by analyzing the number of estimated CPU cycles obtained with \textit{callgrind}.
After that, we showed that the estimates are close to the real values, by comparing them to the time metrics obtained directly from
the processor's registers. The results presented here were obtained on a powerful, modern-day computer. Despite that, they are still
relevant when considering the costs on constrained \gls{iot} devices. While on a different device, the absolute cost 
numbers will be different, they would still maintain a similar proportion one to another and follow a similar trend.
Moreover, the developed tooling can be used to obtain profiling results on any machine, thus giving device-specific 
cost information. The formula used to obtain the CPU cycle count estimate can also be changed to one's needs.

\section{Future Work}

In our work we obtained and analyzed a large number of metrics obtained with \textit{callgrind}. While \textit{callgrind}
provides only an estiamtes of the CPU cycles used, we later showed  that they reflect real values by comparing them with the
time results obtained with \gls{papi}. However, it is important to remember that those mertics were obtained on a general-purpose
computer. While we fixed the CPU frequency and disabled some hardware optimizations, the environment on an \gls{iot} device is still
expected to be very different, due factors such as a lower clock frequency, memory and cache size. Thus, it would be interesting to 
measure time on an gls{iot} device, since the results will differ.

Another characteristic of numerous \gls{iot} devices is limited power (\textit{e.g.} using battery as a power source). 
Thus, it would be interesting analyze the cost of \gls{tls} in terms of power usage. This would also allow to reach 
interesting conclusions, such as: \textit{Using the TLS configuration X would reduce the device's battery life by Y days}.