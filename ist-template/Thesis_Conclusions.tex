%%%%%%%%%%%%%%%%%%%%%%%%%%%%%%%%%%%%%%%%%%%%%%%%%%%%%%%%%%%%%%%%%%%%%%%%
%                                                                      %
%     File: Thesis_Conclusions.tex                                     %
%     Tex Master: Thesis.tex                                           %
%                                                                      %
%     Author: Andre C. Marta                                           %
%     Last modified :  2 Jul 2015                                      %
%                                                                      %
%%%%%%%%%%%%%%%%%%%%%%%%%%%%%%%%%%%%%%%%%%%%%%%%%%%%%%%%%%%%%%%%%%%%%%%%

\chapter{Conclusions and Future Work}
\label{chapter:conclusions}

\section{Conclusions}

This dissertation presented a thorough evaluation of the \gls{tls} protocol, in light of its usage in the \gls{iot} environment. 
We analyzed the  costs of \gls{tls} at the low, middle and the high levels. At the low level, we studied and compared,
the costs of each one of the algorithms that enable the security sevices. We concluded that the choice
of the cheapest algorithm, depended not only on the key size, but also on the peer whose costs we wanted to
minimize. 

At the middle level, we analyzed the costs of the security services of authentication and \gls{pfs}.
We answered the question of how much each security service costs in terms of the number of CPU cycles and time.
The costs of each security service were dissected in light of the results obtained in the analysis of the
individual algorithms at the low level.

At the high level, we analyzed the cost of the Handhshake as a whole. We presented several decision trees that can be
used as a framework to select the cheapest \gls{tls} configuration, according to one's environments needs and limitaitons.
The cost of the Handhshake was dissected in light of the costs of the security services. As a result, we derived a formula
that decomposes the costs of the \gls{tls} Handhshake into individual parts.

While the focus thoroughout this work as on the costs of the Handhshake, we also evaluted the costs of the security services of
confidentiality and itegrity. The asymetric encryption algorithms were profiled in order to answer the question of when the costs
of the Handhshake equate the costs of confidentiality and itegrity. Our analysis showed that, in a typical configuration that is used
on the internet, less than $1.7$MB of data need to be exchanged between the peers in order for that to happen. For this reason, we 
concluded that it only made sense to heavily optimize the Handhshake if the amount of exchanged data is small.

As a result, we presented the most complete and detailed analysis of the costs of the \gls{tls} protocol that exists to date.
The herein presented results can be used by software engineers and security professionals to make informed decisions about the security/cost trade-offs,
specific to the environment.

In the process of the work on this thesis, we not only evaluated \gls{tls} at the level that was never done before, 
but also contributed to the global security community, by:

\begin{enumerate}
    \item contributing to the specification of the \gls{tls} protocol version $1.3$, and to the lesser extent, of \gls{dtls} protocol verison $1.3$
    \item finding and reporting a security vulnerability in \textit{mbedTLS}, which was assigned a \gls{cve} with id \textit{CVE-2018-1000520}
\end{enumerate}

For the evaluation,
we used the \gls{tls} implementaiton of the \textit{mbedTLS} library, which is one of the most
popular \gls{tls} implementaiton libraries for embedded systems. We used two cost metrics
First, we did a thorough analysis of \gls{tls}, by analyzing the number of estimated CPU cycles obtained with \textit{callgrind}.
After that, we showed that the estimates are close to the real values, by comparing them to the time metrics obtained directly from
the processor's registers. The results presented here were obtained on a powerful, modern-day computer. Despite that, they are still
relevant when considering the costs on constrained \gls{iot} devices. While on a different device, the absolute cost 
numbers will be different, they would still maintain a similar proportion one to another and follow a similar trend.
Moreover, the developed tooling can be used to obtain profiling results on any machine, thus giving device-specific 
cost information. The formula used to obtain the CPU cycle count estimate can also be changed to one's needs.

\section{Future Work}

In our work we obtained and analyzed a large number of metrics obtained with \textit{callgrind}. While \textit{callgrind}
provides only an estiamtes of the CPU cycles used, we later showed  that they reflect real values by comparing them with the
time results obtained with \gls{papi}. However, it is important to remember that those mertics were obtained on a general-purpose
computer. While we fixed the CPU frequency and disabled some hardware optimizations, the environment on an \gls{iot} device is still
expected to be very different, due factors such as a lower clock frequency, memory and cache size. Thus, it would be interesting to 
measure time on an gls{iot} device, since the results will differ.

Another characteristic of numerous \gls{iot} devices is limited power (\textit{e.g.} using battery as a power source). 
Thus, it would be interesting analyze the cost of \gls{tls} in terms of power usage. This would also allow to reach 
interesting conclusions, such as: \textit{Using the TLS configuration X would reduce the device's battery life by Y days}.