%%%%%%%%%%%%%%%%%%%%%%%%%%%%%%%%%%%%%%%%%%%%%%%%%%%%%%%%%%%%%%%%%%%%%%%%
%                                                                      %
%     File: Thesis_Resumo.tex                                          %
%     Tex Master: Thesis.tex                                           %
%                                                                      %
%     Author: Andre C. Marta                                           %
%     Last modified :  2 Jul 2015                                      %
%                                                                      %
%%%%%%%%%%%%%%%%%%%%%%%%%%%%%%%%%%%%%%%%%%%%%%%%%%%%%%%%%%%%%%%%%%%%%%%%

\section*{Resumo}

% Add entry in the table of contents as section
\addcontentsline{toc}{section}{Resumo}


\newacronym{idc}{IdC}{Internet das Coisas}
\newacronym{sfp}{SPF}{Segurança Fututra Perfeita}

\textit{Transport Layer Security} (TLS) é um dos protocolos de segurança de comunicação mais
usados no mundo. O seu objectivo principal é fornecer um canal de comunicação seguro
com garantias de confidencialidade, integridade, autenticidade e \gls{sfp}.
Cada um dos seus serviços de segurança pode ser implementado por um dos vários
algoritmos disponíveis. TLS não foi desenhado para o ambiente constrangido
das \gls{idc}, sendo demasiado exigente para tais dispositivos. Contudo, é um
protocolo maleável e os seus serviços de segurança podem sem activados ou
desactivados por conexão. A abdicação de um serviço de segurança ou uso de um
algoritmo menos exigente computacionalmente, reduz os recursos utilizados.
As propriedades de segurança da conexão são definidas pela configuração TLS
em uso. Algumas das configurações podem ser usadas em ambiente das \gls{idc}.
O trabalho existente foca-se no protocolo Datagram TLS e, ou está intimamente
ligado ao uso de um protocolo específico ou requer o uso de uma entidade
externa. Nesta tese nós iremos fazer uma análise completa do protocolo TLS e
dos seus serviços de segurança. Os resultados e diagramas aqui apresentados
podem ser usados pelos programadores e profissionais de segurança para seleccionar
a configuração de TLS mais leve, conforme os requisitos e limitações
do seu ambiente. Iremos avaliar a implementação do protocolo TLS pela
biblioteca \textit{mbedTLS}, usando duas métricas de custo: o número estimado
de ciclos da CPU, obtidos com \textit{valgrind}, e tempo, obtido com PAPI.
De seguida, iremos comprovar que os valores estimados estão perto da realidade.

\vfill

\textbf{\Large Palavras-chave:} TLS, DTLS, SSL, Internet das Coisas, Sistemas Embebidos

