% This is the file where my Master Thesis will be written. It uses the adapted
% LNCS Template.
%
% I'll be using a few codes in the comments, which can be easily looked up:
% * NOTE: theseis-text related comments
% * TODO: thesis-related TODO's
% * WARN: latex/formatting-related warningss
%
% WARN: for running head, subsititute the line below by:
% \documentclass[runningheads,a4paper]{llncs}
\documentclass{llncs}
%
%%% WARN: custom extension
\usepackage{xcolor}
\newcommand{\todo}[1]{\textcolor{red}{TODO: #1}\PackageWarning{TODO:}{#1!}}

%%% GLOSSARIES
\usepackage{glossaries}

%%% inline code
\usepackage{xparse}

\NewDocumentCommand{\codeword}{v}{%
\texttt{\textcolor{blue}{#1}}%
}

\makeglossaries

\newacronym{tls}{TLS}{Transport Layer Security}%
\newacronym{ssl}{SSL}{Secure Sockets Layer}%
\newacronym{ietf}{IETF}{Internet Engineering Task Force}%
\newacronym{mac}{MAC}{Message Authentication Code}%
\newacronym{psk}{PSK}{Pre-Shared Key}%
\newacronym{aead}{AEAD}{Authenticated Encryption With Associated Data}%


%%% WARN: added as specified here:
% https://tex.stackexchange.com/questions/272200/table-of-contents-showing-the-title-as-only-entry-latex
\setcounter{tocdepth}{2}
\makeatletter
\renewcommand*\l@author[2]{} % removes author name from TOC
\renewcommand*\l@title[2]{} % removes title name from TOC
\makeatletter
%%%
%
\usepackage{makeidx}  % allows for indexgeneration
%
\begin{document}
%
\frontmatter          % for the preliminaries
%
\pagestyle{headings}  % switches on printing of running heads
%
\addtocmark{TLS For IoT} % additional mark in the TOC

\tableofcontents
\newpage

\mainmatter              % start of the contributions
%
\title{Transport Layer Security Protocol For Internet Of Things}
%
\titlerunning{TLS For IoT}  % abbreviated title (for running head)
%                                     also used for the TOC unless
%                                     \toctitle is used
%
\author{{Illya Gerasymchuk} \\
\email{illya.gerasymchuk@tecnico.uliboa.pt},\\ WWW home page:
\texttt{https://iluxonchik.github.io/}}
%
\authorrunning{Illya Gerasymchuk} % abbreviated author list (for running head)
%
%%%% list of authors for the TOC (use if author list has to be modified)
\tocauthor{Illya Gerasymchuk}
%
\institute{Instituto Superior Técnico}
% WARN: supper hacked-in
\supervisors{Ricardo Chaves, Aleksandar Ilic}
\maketitle              % typeset the title of the contribution

\begin{abstract}
The abstract should summarize the contents of the paper
using at least 70 and at most 150 words. It will be set in 9-point
font size and be inset 1.0 cm from the right and left margins.
There will be two blank lines before and after the Abstract. \dots
\keywords{TLS, IoT, cryptography, protocol, lightweight cryptography}
\end{abstract}
%
\section{Intoduction}
%
\todo{Intruduce the topic: explain what is IoT; what is TLS; what are the issues with
using RAW TLS with IoT(power, computation, limited resources).}
%
\subsection{Goals}
%
\section{Related Work}
%
\todo{Tell that first I describe the parts of TLS that are common to both and then
specialize for TLS 1.2 and TLS 1.3}
%
\subsection{The TLS Protocol}
TLS stands for Transport Layer Security, it's a \textbf{client-server} protocol
that runs on top a \textbf{connection-oriented and reliable transport protocol},
such as \textbf{TCP}. Its main goal is to provide \textbf{privacy} and \textbf{integrity}
between the two communicating peers. Privacy implies that a third party will not
be able to read the data, while integrity means that a third party will not be
able to alter the data.

In the TCP/IP Protocol Stack, \gls{tls} is placed between the \textbf{Transport}
and \textbf{Application} layers. It's designed to make the application developer's
life easier: all the developer has to do is create a "secure" connection, instead
of a "normal" one.

%%% NOTE: place this in intro? Is this even needed?
%%% NOTE: review, rephrase
%%% NOTE: find better placement
\subsubsection{SSL vs TLS: What's The Difference?}
You will find the names \gls{ssl} and \gls{tls} used interchangeably in the literature,
so I think it's important to distinguish both. \gls{tls} is an evolution of the \gls{ssl} protocol. The protocol changed
its name from \gls{ssl} to \gls{tls} when it was
standardized by the \gls{ietf}.\gls{ssl}
was a proprietary protocol owned by Netscape Communications, and The \gls{ietf}
decided that it was a good idea to standarize it, which resulted in \textbf{RFC 2246},
specifying \gls{tls} 1.0, which was nothing more than a new version \gls{ssl} 3.0,
very few changes were made.
%
% TODO: add some data supporting the TLS 1.2 usage claim
In this document, I'll be concentrating on \gls{tls} 1.2 and \gls{tls} 1.3 protocols.
The first one is the most recently standardized version of \gls{tls} and the latter
is currently and in-draft version with many improvements and optimizations relevant
for the topic of this dissertation.
%

\todo{Explain what RFCs are?}
%
\subsection{Security Services}
%
\gls{tls} provides the following security services:
\begin{itemize}
\item \textbf{authentication} - both, \textbf{peer entity} and \textbf{data origin} (or \textbf{integrity})
authentication.
\subitem \textbf{peer entity authentication} - we can be sure that we’re talking to certain entity, for example, \codeword{www.google.com}.
This is achieved thought the use of \textbf{asymmetric} or \textbf{public key cryptography} (for example, \codeword{RSA} and \codeword{DSA})
or \textbf{symmetric cryptography}, using a \gls{psk}.
\item \textbf{confidentiality} - the data transmitted between the communicating
entities (the client and the server) is encrypted. Symmetric cryptography is
used of data encryption (for exmaple, \codeword{AES}).
\item \textbf{integrity} (also called \textbf{data origin authentication}) - we can be sure that the data was not modified or forged,
\textit{i.e.}, be sure that the data that we’re receiving is coming from the expected entity (for example, we can be sure
that the \codeword{index.html} file sent to us when we connected to \codeword{www.google.com} in fact
came from \codeword{www.google.com} and that it was not modified (i.e tampered with) en
route by an attacker (\textbf{data integrity}). This is achieved through the use
of a keyed \gls{mac} or an \gls{aead} cipher.
\end{itemize}

\subsection{TLS 1.2}
The current version in use is \gls{tls} 1.2.

%
\paragraph{Notes and Comments.}
This is an example of a paragraph. Note the styling.

\subsection{TLS 1.3}
Despite the protocol name not suggesting it \gls{tls} 1.3 is
very different from \gls{tls} 1.2, in fact, it should've probably been called
\gls{tls} 2.0 instead.

%
% ---- Bibliography ----
%
\nocite{*}
\bibliographystyle{splncs03}
\bibliography{tls_for_iot}
%
\printglossary[style=long]
%
\end{document}
